\documentclass{article}
%%%%%%%%%%%%%%%%%%%%%%%%%%%%% Define Article %%%%%%%%%%%%%%%%%%%%%%%%%%%%%%%%%%
%%%%%%%%%%%%%%%%%%%%%%%%%%%%%%%%%%%%%%%%%%%%%%%%%%%%%%%%%%%%%%%%%%%%%%%%%%%%%%%

%%%%%%%%%%%%%%%%%%%%%%%%%%%%% Using Packages %%%%%%%%%%%%%%%%%%%%%%%%%%%%%%%%%%
\usepackage{float}
\usepackage[letterpaper,portrait]{geometry}
\usepackage{graphicx}
\usepackage{anysize}
\usepackage{lipsum}
\usepackage{amsmath,amssymb,amsthm}
\usepackage[utf8]{inputenc}
\usepackage{multirow}
\usepackage{csquotes}
\usepackage[spanish]{babel}
\usepackage{apacite}
\usepackage{multicol}
\usepackage{parskip}
\usepackage{setspace}
\usepackage{empheq}
\usepackage{mdframed}
\usepackage{booktabs}
\usepackage{lipsum}
\usepackage{graphicx}
\usepackage{color}
\usepackage{psfrag}
\usepackage{pgfplots}
\usepackage{bm}
\usepackage{tocloft}
\usepackage{lscape}
\usepackage{adjustbox}
\setlength{\tabcolsep}{1.505625pt}
\renewcommand{\arraystretch}{1.2}
%%%%%%%%%%%%%%%%%%%%%%%%%%%%%%%%%%%%%%%%%%%%%%%%%%%%%%%%%%%%%%%%%%%%%%%%%%%%%%%

% Other Settings

%%%%%%%%%%%%%%%%%%%%%%%%%% Page Setting %%%%%%%%%%%%%%%%%%%%%%%%%%%%%%%%%%%%%%%
\geometry{letterpaper, margin=2.54cm}

%%%%%%%%%%%%%%%%%%%%%%%%%% Define some useful colors %%%%%%%%%%%%%%%%%%%%%%%%%%
\definecolor{ocre}{RGB}{243,102,25}
\definecolor{mygray}{RGB}{243,243,244}
\definecolor{deepGreen}{RGB}{26,111,0}
\definecolor{shallowGreen}{RGB}{235,255,255}
\definecolor{deepBlue}{RGB}{61,124,222}
\definecolor{shallowBlue}{RGB}{235,249,255}
%%%%%%%%%%%%%%%%%%%%%%%%%%%%%%%%%%%%%%%%%%%%%%%%%%%%%%%%%%%%%%%%%%%%%%%%%%%%%%%

%%%%%%%%%%%%%%%%%%%%%%%%%% Define an orangebox command %%%%%%%%%%%%%%%%%%%%%%%%
\newcommand\orangebox[1]{\fcolorbox{ocre}{mygray}{\hspace{1em}#1\hspace{1em}}}
%%%%%%%%%%%%%%%%%%%%%%%%%%%%%%%%%%%%%%%%%%%%%%%%%%%%%%%%%%%%%%%%%%%%%%%%%%%%%%%

%%%%%%%%%%%%%%%%%%%%%%%%%%%% English Environments %%%%%%%%%%%%%%%%%%%%%%%%%%%%%
\newtheoremstyle{mytheoremstyle}{3pt}{3pt}{\normalfont}{0cm}{\rmfamily\bfseries}{}{1em}{{\color{black}\thmname{#1}~\thmnumber{#2}}\thmnote{\,--\,#3}}
\newtheoremstyle{myproblemstyle}{3pt}{3pt}{\normalfont}{0cm}{\rmfamily\bfseries}{}{1em}{{\color{black}\thmname{#1}~\thmnumber{#2}}\thmnote{\,--\,#3}}
\theoremstyle{mytheoremstyle}
\newmdtheoremenv[linewidth=1pt,backgroundcolor=shallowGreen,linecolor=deepGreen,leftmargin=0pt,innerleftmargin=20pt,innerrightmargin=20pt,]{theorem}{Theorem}[section]
\theoremstyle{mytheoremstyle}
\newmdtheoremenv[linewidth=1pt,backgroundcolor=shallowBlue,linecolor=deepBlue,leftmargin=0pt,innerleftmargin=20pt,innerrightmargin=20pt,]{definition}{Definition}[section]
\theoremstyle{myproblemstyle}
\newmdtheoremenv[linecolor=black,leftmargin=0pt,innerleftmargin=10pt,innerrightmargin=10pt,]{problem}{Problem}[section]
%%%%%%%%%%%%%%%%%%%%%%%%%%%%%%%%%%%%%%%%%%%%%%%%%%%%%%%%%%%%%%%%%%%%%%%%%%%%%%%

%%%%%%%%%%%%%%%%%%%%%%%%%%%%%%% Plotting Settings %%%%%%%%%%%%%%%%%%%%%%%%%%%%%
\usepgfplotslibrary{colorbrewer}
\pgfplotsset{width=8cm,compat=1.9}
%%%%%%%%%%%%%%%%%%%%%%%%%%%%%%%%%%%%%%%%%%%%%%%%%%%%%%%%%%%%%%%%%%%%%%%%%%%%%%%

%%%%%%%%%%%%%%%%%%%%%%%%%%%%%%% Title & Author %%%%%%%%%%%%%%%%%%%%%%%%%%%%%%%%
\author{Gustavo Vergara}
%%%%%%%%%%%%%%%%%%%%%%%%%%%%%%%%%%%%%%%%%%%%%%%%%%%%%%%%%%%%%%%%%%%%%%%%%%%%%%%

\begin{document}
\pgfplotsset{compat=1.18}
\setstretch{2}

\begin{titlepage}
	\centering
	\vspace{2.5cm}
	{\scshape \Large TALLER DE METROLOGÍA APLICADA\par}
	\vspace{5cm}
	\textbf\large\scshape{\par}
	\vspace{0.5cm}
	{\Large Vergara Pareja Gustavo\par}
	\vspace{5cm}
	{\scshape\Large Miguel Ángel Lancheros\par}
	\vspace{0.3cm}
	{\scshape\Large Metrología y Control de Calidad - G1IM \par}
	\vspace{0.3cm}
	{\scshape\Large Universidad de Córdoba\par}
	\vspace{0.3cm}
	{\Large 7 de Diciembre de 2023 \par}
\end{titlepage}
\tableofcontents
\newpage
\section{Fabricación de automoviles}
\begin{itemize}
	\item La fabricación de automóviles es un proceso complejo que involucra una gran cantidad de pasos y operaciones. Desde la extracción de materias primas hasta el ensamblaje final, el proceso de fabricación de automóviles debe ser preciso y controlado para garantizar que los productos cumplan con los requisitos de calidad y seguridad.

	El sistema metrológico es un conjunto de principios, definiciones, normas, métodos de medición y aseguramiento de la calidad que se utilizan para garantizar la precisión y la trazabilidad de las mediciones. El sistema metrológico es esencial para la fabricación de automóviles, ya que permite a los fabricantes garantizar que los productos cumplan con las especificaciones y los requisitos de seguridad.
\end{itemize}
\section{Normas que aplican}
Las normas que aplican al sistema metrológico en la fabricación de automóviles incluyen:

	\begin{itemize}
		\item ISO 9001:2015 - Sistema de gestión de la calidad
		\item ISO 14001:2015 - Sistema de gestión ambiental
		\item ISO 45001:2018 - Sistema de gestión de la seguridad y salud ocupacional
		\item IATF 16949:2016 - Sistema de gestión de la calidad para la industria automotriz
	\end{itemize}
	
\section{Equipos de medición}
Los equipos de medición que se utilizan en la fabricación de automóviles incluyen:

\begin{itemize}
	\item Balanzas
	\item Micrómetros
	\item Calibradores
	\item Medidores de tensión (UTM)
	\item Medidores de resistencia
	\item Medidores de temperatura
	\item Medidores de presión
\end{itemize}
\section{Tabla de unidades y conversiones}
La siguiente tabla muestra algunas de las unidades y conversiones que se utilizan en la fabricación de automóviles:
\begin{table}[!ht]
    \centering
    \begin{tabular}{|l|l|l|}
    \hline
        Unidad & Definición & Conversión \\ \hline
        Metro (m) & Unidad básica de longitud & 1 m = 100 cm = 1000 mm = 39,37 in \\ \hline
        Kilogramo (kg) & Unidad básica de masa & 1 kg = 1000 g = 2,2046226218488 lb \\ \hline
        Segundo (s) & Unidad básica de tiempo & 1 s = 1/60 min = 1/3600 h \\ \hline
        Kelvin (K) & Unidad básica de temperatura & 0 °C = 273,15 K \\ \hline
        Pascal (Pa) & Unidad básica de presión & 1 Pa = 1 N/m^{2} \\ \hline
        Voltio (V) & Unidad básica de tensión & 1 V = 1 J/C \\ \hline
        Ohm (\Omega) & Unidad básica de resistencia & 1 \Omega  = 1 V/A \\ \hline
    \end{tabular}
\end{table}
\section{Ejemplo numérico}


Un fabricante de automóviles está fabricando un neumático que debe tener un diámetro de 400,6 mm. El fabricante utiliza una regla para medir el diámetro del neumático y obtiene una lectura de 405 mm. Para verificar la precisión de la medición, el fabricante utiliza un micrómetro y obtiene una lectura de 401 mm.

La diferencia entre las dos mediciones es de 0,4 mm. Esta diferencia es menor que la tolerancia permitida de 1 mm, por lo que el neumático cumple con las especificaciones.

En este ejemplo, el sistema metrológico ha permitido al fabricante garantizar que el neumático cumple con las especificaciones de tamaño.
\bibliographystyle{apacite}
\nocite{*}
\bibliography{cartas-por-atributos}
\end{document}